\abstract{
The TU Delft Astrodynamics Toolbox (Tudat), is an open-source astrodynamics toolbox written in C++ with a focus on numerical state propagation. It has been under development by staff and students of the Astrodynamics and Space Missions (AS) section of Faculty of Aerospace Engineering at Delft University of Technology (TUD) for more than ten years. During the process, Tudat has become a key part of the Space Exploration MSc curriculum at TUD. Moreover, Tudat has been used in dozens of MSc projects, 5 PhD projects, and has contributed to about 20 peer-reviewed publications. \\
Tudat provides both numerical state propagation and state estimation functionalities, organized in a modular setup. The design of Tudat ensures mutual consistency between the various environmental and dynamical models. Further development of Tudat is currently underway or planned in several categories, such as the integration with Machine Learning frameworks, the development of Precise Orbit Determi- nation functionalities with real mission data, and the addition of more mission design and optimization techniques, including the integration with ESA's PaGMO/PyGMO toolbox. \\
Despite a continuous growth seen in the user base, the Tudat developer team has made the decision to create a Python interface for the software, motivated by the necessity of providing a more accessible and user-friendly tool that can be used not only by students and professionals, but also by space enthusiasts.  Indeed, as a modern, high-level programming language, Python provides lower barriers to entry for users, signifcantly increasing the value of the Tudat kernel in education and academia.  Named ``TudatPy'', this project has now become an ongoing effort to improve the accessibility of Tudat and to further develop the Application Programming Interface (API) for integration with external optimization and machine learning frameworks, while fully exploiting the Python language. \\
This paper thus demonstrates the use of the TudatPy API to 1) carry out an arbitrary optimization routine for a spacecraft mission using analytical methods, 2) perform the numerical simulation of the determined mission profile, 3) demonstrate a numerical analysis involving various integration schemes, acceleration models, and propagation methods, and 4) highlight various possibilities for uncertainty analysis and simulated trajectory estimation. \\
As an open-source project, TudatPy, and its installation and documentation, can be found at: \\
\href{https://tudat-space.readthedocs.io/en/latest/}{tudat-space.readthedocs.io}
}
