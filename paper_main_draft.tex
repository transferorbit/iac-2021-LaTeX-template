\documentclass[]{IAC_style_updated}


\begin{document}

\IACpaperyear{2022} % format yyyy
\IACpapernumber{IAC-22,C1,IP,18,x73772} % full paper id
\IAClocation{Paris, France} % used in the header
\IACdate{18-22 September 2022} % used in the header

% Which copyright? If B, put the copyright holder.
\IACcopyrightA{}
%\IACcopyrightB{LateX association}

\title{TudatPy - a versatile open-source astrodynamics software package for space education}

%\IACauthor{Author name}{corresponding affiliation: nr. of corresponding affiliation}{is corresponding author? 0-1}
\IACauthor{Kevin Cowan}{1}{1}
\IACauthor{Filippo Oggioni}{2}{0}
\IACauthor{Geoffrey Garrett}{2}{0}
\IACauthor{Miguel Avillez}{2}{0}
\IACauthor{Joao Encarnacao}{2}{0}
\IACauthor{Marie Fayolle}{2}{0}
\IACauthor{Jonas Hener}{2}{0}
\IACauthor{Marc Naeije}{2}{0}
\IACauthor{Maarten van Nistelrooij}{2}{0}
\IACauthor{Ron Noomen}{2}{0}
\IACauthor{Michael Plumaris}{2}{0}
\IACauthor{Jeremie Gaffarel}{2}{0}
\IACauthor{Sean Cowan}{2}{0}
\IACauthor{Dominic Dirkx}{2}{0}


% Input affiliations here, order is relevant
\IACauthoraffiliation{Space Department, Faculty of Aerospace Engineering, Delft University of Technology, Kluyverweg 1, 2629 HS, Delft, Netherlands \normalfont{E-mail:\ \texttt{k.j.cowan@tudelft.nl}}}
\IACauthoraffiliation{Space Department, Faculty of Aerospace Engineering, Delft University of Technology, Kluyverweg 1, 2629 HS, Delft, Netherlands}


\abstract{
The TU Delft Astrodynamics Toolbox (Tudat), is an open-source astrodynamics toolbox written in C++ with a focus on numerical state propagation. It has been under development by staff and students of the Astrodynamics and Space Missions (AS) section of Faculty of Aerospace Engineering at Delft University of Technology (TUD) for more than ten years. During the process, Tudat has become a key part of the Space Exploration MSc curriculum at TUD. Moreover, TUTudatDAT has been used in dozens of MSc projects, 5 PhD projects, and has contributed to about 20 peer-reviewed publications.    Tudat provides both numerical state propagation and state estimation functionalities, organized in a modular setup. The design of Tudat ensures mutual consistency between the various environmental and dynamical models. Further development of Tudat is currently underway or planned in several categories, such as the integration with Machine Learning frameworks, the development of Precise Orbit Determi- nation functionalities with real mission data, and the addition of more mission design and optimization techniques, including the integration with ESA's PaGMO/PyGMO toolbox.    Despite a continuous growth seen in the user base, the Tudat developer team has made the decision to create a Python interface for the software, motivated by the necessity of providing a more accessible and user-friendly tool that can be used not only by students and professionals, but also by space enthusiasts.  Indeed, as a modern, high-level programming language, Python provides lower barriers to entry for users, signi cantly increasing the value of the Tudat kernel in education and academia.  Named "TudatPy", this project has now become an ongoing effort to improve the accessibility of Tudat and to further develop the Application Programming Interface (API) for integration with external optimization and machine learning frameworks, while fully exploiting the Python language.  This paper thus demonstrates the use of the TudatPy API to 1) carry out an arbitrary optimization routine for a spacecraft mission using analytical methods, 2) perform the numerical simulation of the determined mission profile, 3) demonstrate a numerical analysis involving various integration schemes, acceleration models, and propagation methods, and 4) highlight various possibilities for uncertainty analysis and simulated trajectory estimation.    As an open-source project, TudatPy, and its installation and documentation, can be found at https: tudat-space.readthedocs.io/en/latest/ 

}


\maketitle
\thispagestyle{fancy} % resets proper header/footer

\textcolor{xkcdRed}{DO NOT EDIT THIS MAIN FILE. TO ADD YOUR MATERIAL, PLACE IT IN A SEPARATE .TEX FILE, WHICH WILL THEN BE IMPORTED USING AN $\mathtt{\backslash INPUT\lbrace\rbrace}$ STATEMENT.}

\section*{Nomenclature}
    \lbrack Input required here. \rbrack

\section*{Acronyms/Abbreviations}
    %\input{acronyms_abbreviations}
    \lbrack Input required here. \rbrack

\section{What is TudatPy?}
    %\lbrack Target size of text in this section: ½ an A4\rbrack

\lbrack Input required here.\rbrack

What can you do with Tudat? \\
Consider the following:
\begin{itemize}
    \item What are the high-level areas in which TudatPy can be applied? 
    \item From the point of view of a user, what can you do with it?
\end{itemize}
    
The primary user, for the purposes of this paper, is assumed to be: someone interested in using TudatPy in a way similar to that of a TU Delft Space instructor, so for education which has a strong basis in and connection to research.

\fboxrule = 1pt
\fcolorbox{xkcdBlue}{xkcdWhite}{
    \begin{minipage}[]{.9\linewidth}
        Guidance to tudat-team authors: \\
        To do: 
        \begin{list}{$\square$}
            \item As features are discussed, list published work (or educational settings) in which the various features have been applied.
        \end{list}
    \end{minipage} 
    }

\lbrack Input required here.\rbrack

    \lbrack Input required here.\rbrack

\section{What are its key features?}
    %\input{key_features}

    From the point of view of a user, what can you do with it?
    The primary user, for the purposes of this paper, is assumed to be:
    - Someone interested in using TudatPy in a way similar to that of a TU Delft Space instructor, so for education with a strong basis in research.

    Features to mention:

    Advantages / USPs to highlight:
    \begin{itemize}
        \item Modularity
        \item Fidelity
        \item Usability, particularly with the Python interface as well as how this allows the use of the vast range of Python tools now available (eg. Numpy, SciPy, Matplotlib, scikit-learn, ...)
    \end{itemize}

    To do:
    \lbrack \rbrack As features are discussed, list published work (or educational settings) in which the various features have been applied.

    \lbrack Input required here.\rbrack

\section{Example applications}
    %\input{example_applications}
    \lbrack Input required here.\rbrack

    \subsection{Analytical analyses}
    %\input{analytical_analyses}
    Proposal: \\
    Two-part presentation: 
    \begin{enumerate}
        \item Single arc propagation
        \item MGA-DSM
    \end{enumerate}
    \lbrack Input required here.\rbrack

    \subsection{State propagation}
    %\input{state_propagation}
    \lbrack Input required here.\rbrack

    \subsection{State estimation}
    %\lbrack Target size of text in this section: one A4\rbrack

\lbrack Input required here.\rbrack


    \lbrack Input required here.\rbrack

    \subsection{Trajectory Optimization}
    %\lbrack Target size of text in this section: one A4\rbrack

\lbrack Input required here.\rbrack

\fboxrule = 1pt
\fcolorbox{xkcdBlue}{xkcdWhite}{
    \begin{minipage}[]{.9\linewidth}
            Guidance to tudat-team authors: \\
        Without extensively advertizing for PyGMO, mention the integration with PyGMO and the key advantages of integrating it into TudatPy, such as the open-source nature, successful track-record, ..., of PyGMO.
    \end{minipage} 
    }

    \lbrack Input required here.\rbrack
    
    - Without extensively advertizing for PaGMO \lbrack or/PyGMO\rbrack, mention the integration with PaGMO and the key advantages of its integration, such as its open-source nature, successful track-record, ...

\section{Get started with Tudat}
    A brief indication of where to find it, how to get started ``playing'' with it, how to contribute to development, ...

\end{document}
