\documentclass[]{IAC_style}


\begin{document}

\IACpaperyear{2022} % format yyyy
\IACpapernumber{IAC-22,C1,IP,18,x73772} % full paper id
\IAClocation{Paris, France} % used in the header
\IACdate{18-22 September 2022} % used in the header

% Which copyright? If B, put the copyright holder.
\IACcopyrightA{}
%\IACcopyrightB{LateX association}

\title{TudatPy - a versatile open-source astrodynamics software package for space education}

%\IACauthor{Author name}{corresponding affiliation: nr. of corresponding affiliation}{is corresponding author? 0-1}
\IACauthor{Kevin Cowan}{1}{1}
%\IACauthor{Kevin Cowan$^{\orcidlink{0000-0000-0000-0000}}$}{1}{1}
\IACauthor{Filippo Oggioni}{1}{0}
\IACauthor{Geoffrey Garrett}{1}{0}
\IACauthor{Miguel Avillez}{1}{0}
\IACauthor{Joao Encarnacao}{1}{0}
\IACauthor{Marie Fayolle}{1}{0}
\IACauthor{Jonas Hener}{1}{0}
\IACauthor{Marc Naeije}{1}{0}
\IACauthor{Maarten van Nistelrooij}{1}{0}
\IACauthor{Ron Noomen}{1}{0}
\IACauthor{Michael Plumaris}{1}{0}
\IACauthor{Jeremie Gaffarel}{1}{0}
\IACauthor{Sean Cowan}{1}{0}
\IACauthor{Dominic Dirkx}{1}{0}

% Input affiliations here, order is relevant
\IACauthoraffiliation{Space Department, Faculty of Aerospace Engineering, Delft University of Technology, Kluyverweg 1, 2629 HS, Delft, Netherlands \normalfont{E-mail:\ \texttt{\{k.j.cowan,g.garrett,many.more,many.more,many.more,many.more,many.more,many.more,many.more,many.more,many.more,this_many.more,d.dirkx\}@tudelft.nl}}}
%\IACauthoraffiliation{Lorem Ipsum2. Sed ut perspiciatis unde omnis iste natus error sit voluptatem accusantium doloremque laudantium \normalfont{E-mail:\ \texttt{geoffrey.garrett}}}

\abstract{
Lorem ipsum dolor sit amet, consectetur adipiscing elit. Integer vestibulum, libero id tempor suscipit, leo mi viverra nulla, non accumsan nulla nunc sed diam. \\
Here's a new paragraph in the abstract.

%
%
%
}


\maketitle
\thispagestyle{fancy} % resets proper header/footer

\section*{Nomenclature}
    %\input*{nomenclature}
    $\left[Input required here.\right]$

\section*{Acronyms/Abbreviations}
    %    \lbrack Input required here.\rbrack


    $\left[Input required here.\right]$

\section{Introduction}
    %\input{introduction}
    $\left[Input required here.\right]$

\section{What is TudatPy?}
    %    \lbrack Input required here.\rbrack


    $\left[Input required here.\right]$

\section{What are its key features?}
    %    \lbrack Input required here.\rbrack



    From the point of view of a user, what can you do with it?
    The primary user, for the purposes of this paper, is assumed to be:
    - Someone interested in using TudatPy in a way similar to that of a TU Delft Space instructor, so for education with a strong basis in research.

    Features to mention:

    Advantages / USPs to highlight:
    - Modularity
    - Fidelity
    - Usability, particularly with the Python interface as well as how this allows the use of the vast range of Python tools now available (eg. Numpy, SciPy, Matplotlib, scikit-learn, ...)

    To do:
    \left[ \right] As features are discussed, list published work (or educational settings) in which the various features have been applied.

    \left[Input required here.\right]

\section{Example applications}
    %\input{example_applications}
    \left[Input required here.\right]

    \subsection{Analytical analyses}
    %    \lbrack Input required here.\rbrack


    Proposal: Two-part presentation:
    - (1) Single arc propagation
    - (2) MGA-DSM
    \left[Input required here.\right]

    \subsection{State propagation}
    %    \lbrack Input required here.\rbrack


    \left[Input required here.\right]

    \subsection{State estimation}
    %    \lbrack Input required here.\rbrack


    \left[Input required here.\right]

    \subsection{Trajectory Optimization}
    %    \lbrack Input required here.\rbrack


    \left[Input required here.\right]
    
    - Without extensively advertizing for PaGMO/PyGMO, mention the integration with PaGMO/PyGMO and the key advantages of its integration, such as its open-source nature, successful track-record, ...

\end{document}
