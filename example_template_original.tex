\documentclass[]{IAC_style_template_original}


\begin{document}

\IACpaperyear{2022} % format yyyy
\IACpapernumber{IAC-22,C1,IP,18,x73772} % full paper id
\IAClocation{Paris, France} % used in the header
\IACdate{18-22 September 2022} % used in the header

% Which copyright? If B, put the copyright holder.
\IACcopyrightA{}
%\IACcopyrightB{LateX association}

\title{TudatPy - a versatile open-source astrodynamics software package for space education}

%\IACauthor{Author name}{corresponding affiliation: nr. of corresponding affiliation}{is corresponding author? 0-1}
\IACauthor{Kevin Cowan}{1}{1}
\IACauthor{Filippo Oggioni}{2}{0}
\IACauthor{Geoffrey Garrett}{2}{0}
\IACauthor{Miguel Avillez}{2}{0}
\IACauthor{Joao Encarnacao}{2}{0}
\IACauthor{Marie Fayolle}{2}{0}
\IACauthor{Jonas Hener}{2}{0}
\IACauthor{Marc Naeije}{2}{0}
\IACauthor{Maarten van Nistelrooij}{2}{0}
\IACauthor{Ron Noomen}{2}{0}
\IACauthor{Michael Plumaris}{2}{0}
\IACauthor{Jeremie Gaffarel}{2}{0}
\IACauthor{Sean Cowan}{2}{0}
\IACauthor{Dominic Dirkx}{2}{0}


% Input affiliations here, order is relevant
\IACauthoraffiliation{Space Department, Faculty of Aerospace Engineering, Delft University of Technology, Kluyverweg 1, 2629 HS, Delft, Netherlands \normalfont{E-mail:\ \texttt{k.j.cowan@tudelft.nl}}}
\IACauthoraffiliation{Space Department, Faculty of Aerospace Engineering, Delft University of Technology, Kluyverweg 1, 2629 HS, Delft, Netherlands}


\abstract{
Lorem ipsum dolor sit amet, consectetur adipiscing elit. Integer vestibulum, libero id tempor suscipit, leo mi viverra nulla, non accumsan nulla nunc sed diam. \\
Here's a new paragraph in the abstract.

%
%
%
}


\maketitle
\thispagestyle{fancy} % resets proper header/footer

\textcolor{xkcdRed}{DO NOT EDIT THIS MAIN FILE. TO ADD YOUR MATERIAL, PLACE IT IN A SEPARATE .TEX FILE, WHICH WILL THEN BE IMPORTED USING AN $\mathtt{\backslash INPUT\lbrace\rbrace}$ STATEMENT.}

\section*{Nomenclature}
    \lbrack Input required here. \rbrack

\section*{Acronyms/Abbreviations}
    %    \lbrack Input required here.\rbrack


    \lbrack Input required here. \rbrack

\section{What is TudatPy?}
    %    \lbrack Input required here.\rbrack


    \lbrack Input required here.\rbrack

\section{What are its key features?}
    %    \lbrack Input required here.\rbrack



    From the point of view of a user, what can you do with it?
    The primary user, for the purposes of this paper, is assumed to be:
    - Someone interested in using TudatPy in a way similar to that of a TU Delft Space instructor, so for education with a strong basis in research.

    Features to mention:

    Advantages / USPs to highlight:
    \begin{itemize}
        \item Modularity
        \item Fidelity
        \item Usability, particularly with the Python interface as well as how this allows the use of the vast range of Python tools now available (eg. Numpy, SciPy, Matplotlib, scikit-learn, ...)
    \end{itemize}

    To do:
    \lbrack \rbrack As features are discussed, list published work (or educational settings) in which the various features have been applied.

    \lbrack Input required here.\rbrack

\section{Example applications}
    %\input{example_applications}
    \lbrack Input required here.\rbrack

    \subsection{Analytical analyses}
    %    \lbrack Input required here.\rbrack


    Proposal: \\
    Two-part presentation: 
    \begin{enumerate}
        \item Single arc propagation
        \item MGA-DSM
    \end{enumerate}
    \lbrack Input required here.\rbrack

    \subsection{State propagation}
    %    \lbrack Input required here.\rbrack


    \lbrack Input required here.\rbrack

    \subsection{State estimation}
    %    \lbrack Input required here.\rbrack


    \lbrack Input required here.\rbrack

    \subsection{Trajectory Optimization}
    %    \lbrack Input required here.\rbrack


    \lbrack Input required here.\rbrack
    
    - Without extensively advertizing for PaGMO \lbrack or/PyGMO\rbrack, mention the integration with PaGMO and the key advantages of its integration, such as its open-source nature, successful track-record, ...

\section{Get started with Tudat}
    A brief indication of where to find it, how to get started ``playing'' with it, how to contribute to development, ...

\end{document}
